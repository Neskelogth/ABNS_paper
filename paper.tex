\documentclass[conference]{IEEEtran}
\IEEEoverridecommandlockouts
% The preceding line is only needed to identify funding in the first footnote. If that is unneeded, please comment it out.
\usepackage{cite}
\usepackage{amsmath,amssymb,amsfonts}
\usepackage{algorithmic}
\usepackage{graphicx}
\usepackage{textcomp}
\usepackage{xcolor}
%\usepackage{bibtex}
\def\BibTeX{{\rm B\kern-.05em{\sc i\kern-.025em b}\kern-.08em
    T\kern-.1667em\lower.7ex\hbox{E}\kern-.125emX}}
\begin{document}

\title{Proposition of a New Experiment to Better Understand the Relation Between Typicality and Prototypes}



\author{\IEEEauthorblockN{Samuel Kostadinov}
%\IEEEauthorblockA{\textit{dept. name of organization (of Aff.)} \\
\textit{University of Trento}\\
Trento, Italy \\
samuel.kostadinov@unitn.it}


\maketitle

\begin{abstract}

	

\end{abstract}

\begin{IEEEkeywords}
Typicality, Prototypes, CNNs, Siamese Network
\end{IEEEkeywords}










\section{Introduction}
	
	The typicality of a concept is a topic that needs a lot of exploration, since it's difficult to evaluate precisely the typicality of an object 
	and also because the people's brains are always a little different from each other. The experiment I would like to propose, has the objective of 
	make sure we can understand better the bond between the perceived typicality of an object that belongs to a category and the prototype of that category.
	This can help the scientists to better understand how knowledge is organized in the brain. This experiment consists in different phases, such as
	data collection, feature extraction, prototype construction and similarity judgment.



%\section{Related Work}

%	In this section we will cover some related work to the topics that are of interest for this paper. In the first section there is an overview on 
%	feature extraction methods, in the second section there are prototype construction related work, and in the final section there is an 
%	outline of similarity judgment related work. 
	
%	\subsection{Feature extraction related work}\label{fes}	
%	\subsection{Prototype construction related work}\label{pcs}
%	\subsection{Similarity judgment related work}\label{sjs}



\section{Background}

	\subsection{Feature extraction}

		Feature extraction is a topic widely studied. 
		This topic has numerous different applications. 
		One of its goals is to reduce the amount of computational power needed for image processing. 
		There are various techniques to extract meaningful features from images. 
		Some of them are very common and very easy to implement. 
		These are, for example, edge extraction or shape analysis. 
		Other possibilities, more advanced, involve neural networks.\\

		\begin{itemize}
			
			\item \textbf{Edge detection}:\\				
				In 2019, a paper by Owotogbe presented a review of edge detection techniques. 
				These are usually divided into two groups, gradient-based and gaussian-based. 
				Some examples are the Sobel operator and the Canny edge detector. 
				Each of the methods has pros and cons. 
				It's the user's job to find the most appropriate for its goal~\cite{1}.
			
		\end{itemize}

\section{Data Collection}
\section{Feature Extraction}
\section{Prototype Construction}
\section{Similarity Judgment}


\section{Conclusion}
\section{Possible Further Experiments}
%\subsection{Figures and Tables}
%\paragraph{Positioning Figures and Tables} Place figures and tables at the top and 
%bottom of columns. Avoid placing them in the middle of columns. Large 
%figures and tables may span across both columns. Figure captions should be 
%below the figures; table heads should appear above the tables. Insert 
%figures and tables after they are cited in the text. Use the abbreviation 
%``Fig.~\ref{fig}'', even at the beginning of a sentence.
%
%\begin{table}[htbp]
%\caption{Table Type Styles}
%\begin{center}
%\begin{tabular}{|c|c|c|c|}
%\hline
%\textbf{Table}&\multicolumn{3}{|c|}{\textbf{Table Column Head}} \\
%\cline{2-4} 
%\textbf{Head} & \textbf{\textit{Table column subhead}}& \textbf{\textit{Subhead}}& \textbf{\textit{Subhead}} \\
%\hline
%copy& More table copy$^{\mathrm{a}}$& &  \\
%\hline
%\multicolumn{4}{l}{$^{\mathrm{a}}$Sample of a Table footnote.}
%\end{tabular}
%\label{tab1}
%\end{center}
%\end{table}
%
%\begin{figure}[htbp]
%\centerline{\includegraphics{fig1.png}}
%\caption{Example of a figure caption.}
%\label{fig}
%\end{figure}
%
%Figure Labels: Use 8 point Times New Roman for Figure labels. Use words 
%rather than symbols or abbreviations when writing Figure axis labels to 
%avoid confusing the reader. As an example, write the quantity 
%``Magnetization'', or ``Magnetization, M'', not just ``M''. If including 
%units in the label, present them within parentheses. Do not label axes only 
%with units. In the example, write ``Magnetization (A/m)'' or ``Magnetization 
%\{A[m(1)]\}'', not just ``A/m''. Do not label axes with a ratio of 
%quantities and units. For example, write ``Temperature (K)'', not 
%``Temperature/K''.





%\section*{References}
%
%Please number citations consecutively within brackets \cite{b1}. The 
%sentence punctuation follows the bracket \cite{b2}. Refer simply to the reference 
%number, as in \cite{b3}---do not use ``Ref. \cite{b3}'' or ``reference \cite{b3}'' except at 
%the beginning of a sentence: ``Reference \cite{b3} was the first $\ldots$''
%
%Number footnotes separately in superscripts. Place the actual footnote at 
%the bottom of the column in which it was cited. Do not put footnotes in the 
%abstract or reference list. Use letters for table footnotes.
%
%Unless there are six authors or more give all authors' names; do not use 
%``et al.''. Papers that have not been published, even if they have been 
%submitted for publication, should be cited as ``unpublished'' \cite{b4}. Papers 
%that have been accepted for publication should be cited as ``in press'' \cite{b5}. 
%Capitalize only the first word in a paper title, except for proper nouns and 
%element symbols.
%
%For papers published in translation journals, please give the English 
%citation first, followed by the original foreign-language citation \cite{b6}.





\begin{thebibliography}{00}

\bibitem{1} J. S. Owotogbe,T. S. Ibiyemi, and B. A. Adu, ``Edge Detection Techniques on Digital Images - A Review'', International Journal of Innovative Science and Research Technology, vol. 4, issue 11, Nov. 2019

%\bibitem{b1} J. S. Owotogbe,T. S. Ibiyemi, and B. A. Adu, ``Edge Detection Techniques on Digital Images - A Review'', International Journal of Innovative Science and Research Technology, vol. 4, pp. 11, November 2019.
%\bibitem{1} J. Clerk Maxwell, A Treatise on Electricity and Magnetism, 3rd ed., vol. 2. Oxford: Clarendon, 1892, pp.68--73.
%\bibitem{b3} I. S. Jacobs and C. P. Bean, ``Fine particles, thin films and exchange anisotropy,'' in Magnetism, vol. III, G. T. Rado and H. Suhl, Eds. New York: Academic, 1963, pp. 271--350.
%\bibitem{b4} K. Elissa, ``Title of paper if known,'' unpublished.
%\bibitem{b5} R. Nicole, ``Title of paper with only first word capitalized,'' J. Name Stand. Abbrev., in press.
%\bibitem{b6} Y. Yorozu, M. Hirano, K. Oka, and Y. Tagawa, ``Electron spectroscopy studies on magneto-optical media and plastic substrate interface,'' IEEE Transl. J. Magn. Japan, vol. 2, pp. 740--741, August 1987 [Digests 9th Annual Conf. Magnetics Japan, p. 301, 1982].
%\bibitem{b7} M. Young, The Technical Writer's Handbook. Mill Valley, CA: University Science, 1989.
\end{thebibliography}






%\vspace{12pt}
%\color{red}
%IEEE conference templates contain guidance text for composing and formatting conference papers. Please ensure that all template text is removed from your conference paper prior to submission to the conference. Failure to remove the template text from your paper may result in your paper not being published.



\end{document}